\documentclass{article}
\usepackage[utf8]{inputenc}
\usepackage[spanish]{babel}
\usepackage{hyperref}
\usepackage{url}

\title{¿Como la crisis de los fundamentos derivó en el nacimiento de la computación moderna?}
\author{David Esteban Monsalve Pulgarin}
\date{\today}

\usepackage{natbib}
\usepackage{graphicx}

\begin{document}

\maketitle

\section{Introduction}

¿Es posible demostrar si un problema matemático tiene solución?, ¿Puedo saber cuanto tiempo tardare en calcular el resultado de dicho problema? Preguntas como estas fueron las que dieron paso a lo que hoy en día conocemos como la informática moderna, y es que gracias a una crisis de fundamentos que se presentaron en el área de las rigurosas matemáticas nombres como los de Alan Turing y Kurt Gödel pusieron a temblar una poderosa rama de la ciencia que fue protagonista de un gran salto en esa época y es que no era para menos estamos hablando que la matemática y la ingeniería fueron herramientas indispensables para la modificación de la sociedad pues marcaron la revolución industrial. 
El siguiente ensayo tiene como propósito hablar de la historia del computador además de  analizar esa falta de fundamentos , esas famosas paradojas , las teorías de la oposición al movimiento “Formalista” tales como “El teorema de Gödel”, la solución al “Entscheidungsproblem” por parte de Alan Turing y partiendo desde ahí a otro nuevo paso de la informática moderna pues se paso de mecanizar los cálculos a el desarrollo maquinas programables que solo almacenaban datos, hasta un concepto que revoluciono la historia “El computador de programa almacenado”. El ensayo se dirige a 2 principales ramas, ¿Qué es un ordenador y como estos han evolucionado cambiando nuestras vidas?, y ¿Qué es la máquina de Turing y el teorema de Gödel?.


\section{Desarrollo}
\subsection{¿Qué es un ordenador y como estos han evolucionado cambiando nuestras vidas?}

\par 
\vspace{2mm}
La máquina que más límites ha hecho saltar a la humanidad es el computador y es que no es para menos este ha influido  cotidianamente en nuestras vidas, desde labores tan simples como imprimir en consola “Hola mundo”, hasta simular sistemas cuánticos incluso labores como el ensayo que estoy digitando en este momento, se puede decir que es una tendencia que cada vez se proyecta mas en el futuro pues continuamente los avances en informática han sido uno de los pilares en la tecnología moderna, pues la mayor parte los dispositivos en los que trabajas como smartphones, tabletas, Laptops manejan el concepto de “Programa almacenado”, Ósea no solo almacenar datos sino trabajar con esos datos lo que ha dado pie a ramas como la programación de ser casi que un lenguaje universal de esta época.

\par 
\vspace{4mm}

Concretamente ¿Qué es un computador? Es una maquina que puede ser programada para aceptar datos las “Entradas”, tiene la capacidad para procesarlos, produce una información con esos datos procesados “Las salidas” y luego las almacena para un uso posterior. Su ciclo de procesamiento de la información se confina en las siguientes funciones: Aceptar entradas, Procesar datos, almacenar datos e información y por último producir una salida. ¿Simple verdad? Le das una información y este te puede dar solución a cualquier problema (Algoritmo) por medio del proceso de los datos o al menos así lo pensó Alan Turing con su maquina universal capaz de ejecutar cualquier conjunto de instrucciones tardando mas o tardando menos. 

\par 
\vspace{4mm}

El aprendizaje de los computadores nos ayudara a entender mejor el mundo moderno, pero no está de más conocer sus origines y es que la informática tiene una historia bastante larga partiendo desde las maquinas de calcular de Pascal o Leibniz, pasando por el Analytical Engine de Babbage hasta los dispositivos más modernos como los laptops y los smartphones, se inicializo con artefactos que mecanizaban los cálculos un ejemplo de esto fue cuando oleadas masivas de inmigrantes amenazaban con sobrecargar el censo americano de 1890, fue cuando el inspector German Hollerith introdujo un sistema de tarjetas en el que cada persona perforaría las tarjetas en los campos correspondientes a la edad, sexo , raza y profesión, estas funcionaban con unas clavijas eléctricas y almacenaban la información en una  serie de diales como si fueran relojes en otras máquinas que separaban las tarjetas, con este proceso el censo de 1890 de 62 millones de personas se hizo en solo 6 semanas, en mi opinión el hecho de que se tuviera la posibilidad de mecanizar la información era un avance bastante útil pues reducía considerablemente el tiempo en el que se realiza una tarea, luego de esto se pasó a la manipulación de esta información con maquinas programables y por ultimo al concepto de computador de programa almacenado donde no solo se guardan los datos sino los propios programas que ejecutan esa información procesada. 

\par 
\vspace{4mm}

Pensemos en lo siguiente cada hombre, mujer y niño sobre la faz de la tierra deben sumar un aproximado a 2000 números en un segundo para realizar una tarea en específico, Bastante engorroso ¿Verdad? Pues esta tarea es algo sencillo para la maquina Asci White una poderosa computadora capaz de realizar un total de 12 trillones de operaciones matemáticas por segundo, para que te des una idea de lo útil que resulta procesar datos rápidamente piensa en que en 1992 se prohibieron las pruebas nucleares bajo tierra, pero gracias a este dispositivo estas se podrían simular, pero incluso con este poderoso ritmo de trabajo se llevan semanas en completar una sola prueba nuclear. Respecto a esto el dato habla por si solo , un paso que seguramente nos muestra la cantidad de limites que como humanidad hemos sobrepasado, pero realmente ¿cuál es el principio de todo esto? ósea como gracias a el calculo masivo de operaciones matemáticas podemos dar solución a este tipo de problemas modernos, en la actualidad tenemos un sinfín de problemas que nos planteamos a diario pues la ciencia esta en una constante búsqueda de encontrar esas soluciones palabras mas palabras menos ¿Todos los problemas tienen una solución mientras puedan ser descritos en una forma matemática?. 

\par 
\vspace{4mm}
\subsection{¿Qué es el teorema de Gödel y la máquina de Turing?}
\par
¿Alguna vez has tenido un problema en tu vida? Ciertamente si hasta yo los tengo, y cuando los cuentas a otra persona este individuo te dice: - “Tranquilo todos los problemas tienen solución” sin embargo, realmente ¿Todos los problemas tienen solución? Esta duda no es algo nuevo pues esta misma ha estado la mente de personas muy brillantes durante mucho tiempo, pero en mi opinión cada persona tiene un planteamiento distinto de lo que considere como una solución a un problema. 
Ciertamente las matemáticas desde hace mucho tiempo han fundamentado diversas soluciones a muchas incógnitas que el ser humano ha tenido desde sus origines o es que acaso contar, medir, calcular, y entre muchas cosas mas no han sido aportes significativos, ciertamente sí. Ramas de la ciencia como la física en concreto la mecánica clásica nos han permitido modelar a la naturaleza mediante modelos matemáticos, esto quiere decir que cualquier interacción con nuestro mundo puede ser descrita en matemáticas por ejemplo la rotación de nuestro planeta debido a la ley de gravitación universal, el movimiento de los cuerpos, la cinemática y muchos aportes más. Pero entonces ya se daba por hecho que todos los problemas matemáticos tienen una solución tarde o temprano siguiendo una serie de pasos estipulados, esto significa que todos los problemas que puedan ser descritos con matemáticas tienen una solución, Esta afirmación no era del todo cierta y por eso en 1928 David Hilbert formulo el “Entscheidungsproblem” o problema de decisión que en palabras simples decía, ¿Puede existir una serie de pasos ordenados que me diga si la solución que tengo para un problema es verdadero o falso? Pues Hilbert le apuntaba a construir un sistema de símbolos y reglas que demostrara todas las afirmaciones verdaderas, según Hilbert una demostración es la organización de los símbolos a partir de reglas precisas y en un numero finito de pasos hasta formar las expresiones correctas, lo que Hilbert quería responder eran las siguientes preguntas: 

\par
\vspace{4mm}
\begin{enumerate}
    \item [*]¿Son las matemáticas completas? Ósea cualquier proposición puede ser probada o rechazada 
    \item [*]¿Consistentes? Es decir, no es posible demostrar algo falso 
    \item [*]¿Son las matemáticas Fintarías? Que cualquier proposición puede demostrarse como cierta o falsa en una secuencia finita de pasos.
\end{enumerate}

\par
Los teoremas de incompletitud de Gödel rechazaron estas primeras 2 preguntas ya que al auto referenciar un sistema definido de tal forma que en el no haya contradicciones, no se podrá demostrar su falsedad ni su veracidad ya que generaría una inconsistencia y rompería una de las primeras reglas de Hilbert. Muy astuto verdad pues era claro que no se podrían reducir conceptos tan paradójicos como el infinito en algo tan reducido. 
\par
\vspace{2mm}
Desde mi punto de vista si desde la base fundamental se presentan estos fallos estamos diciendo que este lenguaje matemático se convertiría en algo ambiguo ya que la idea partía en que no se generan contradicciones en los sistemas planteados. Gödel decía que, para cualquier conjunto de axiomas, siempre es posible hacer enunciados que a partir de esos axiomas no puede demostrarse que son verdaderos ni falsos por lo que la idea de un sistema matemático completo era un sueño, Pero esto no quiere decir que no podemos encontrar la verdad pues el hecho que un sistema matemático no sea completo no quiere decir que este sea falso el sistema puede seguir siendo muy útil. 
\par
\vspace{2mm}
Partiendo de esto para salvar el “El problema de decisión” quedaba la pregunta de qué ¿Son las matemáticas decidibles? Ósea que esta se demuestre como verdadera o falta tras una secuencia finita de pasos. Si se respondía como afirmativa esta pregunta existirían 3 tipos de proposiciones matemáticas, las que son ciertas o proposiciones simples, las compuestas o falsas donde usamos las negaciones y por ultimo las que no entran en ninguna de estas 2 pero sabemos que existen pero no se sabe donde ubicarlas lo cual es muy diferente a decir que son falsas , verdaderas o que simplemente no existan y es aquí donde Alan Turing toma gran protagonismo pues decide resolver por su cuenta el problema de decisión, ideo un dispositivo mecánico capaz de que por medio de operaciones aritméticas buscara solución a problemas matemáticos, esta maquina consta de una cinta tan larga como se deseara y consistía en un cabezal que se desplazaba hacia la derecha o hacia la izquierda según el dato que interpretaba el cabezal y dependiendo de este imprimía algo y cambiaba al siguiente estado  algo bastante similar a lo que hace un computador donde analógicamente la cinta simula la memoria y el cabezal simula el microprocesador.
\par
\vspace{2mm}
Idealizamos en que para un problema debe haber una única verdad y esta debe ser absoluta, entonces nos limitamos a reducir los conceptos a Una verdad irrefutable un axioma por ejemplo y nos cerramos a las paradojas y a otros fundamentos que nos dicen que no es posible modelar algo tan complejo como la misma naturaleza en algo tan reducido como una sola verdad absoluta a lo que me refiero es que hay muchas posibilidades, al final la máquina de Turing es el modelo mas simple de una computadora, te imaginas una computadora buscando una solución de un problema que no tiene respuesta y que se quedara congelada por toda la eternidad , más allá de que algo tenga solución o no lo importante es que tengas condiciones para intentarlo y saber cuándo detenerse. 

\section{Conclusiones}
\begin{enumerate}
    \item [-] No existen verdades absolutas, el hecho que un sistema matemático no fuera completo ósea que no se pudiera demostrar que era cierta o falsa, no quiere decir que ese sistema matemático sea falso pues ese sistema podría resultar bastante útil, por ejemplo, cuando demuestras algo por reducción al absurdo estas partiendo de algo que no necesariamente es falso ni verdadero de hecho es algo inconsistente, pero puedes probar que un ejercicio o una propiedad es verdadera o falsa gracias a esto.
    \item [-]Los teoremas de incompletitud de Gödel rechazaron estas primeras 2 preguntas ya que al auto referenciar un sistema definido de tal forma que en el no haya contradicciones, no se podrá demostrar su falsedad ni su veracidad ya que generaría una inconsistencia y rompería una de las primeras reglas de Hilbert.
    \item [-]Alan Turing al deducir una respuesta para los problemas matemáticos por medio de una secuencia de cambios de estados, o pasos dio pie para que independientemente el tiempo en el que se calculara si un problema era resoluble o no, se planteara una idea de cálculos infinitos los cuales rompían con varias paradoras y serviría como una de mayores inspiraciones a la informática moderna. 
\end{enumerate}
\section{Cibergrafias}
\url{https://www.youtube.com/watch?v=h1wRsbwnvDI&t=554s}
\par
\vspace{2mm}
\url{https://www.monografias.com/trabajos15/historia-computador/historia-computador}
\par
\vspace{2mm}
\url{https://www.youtube.com/watch?v=Es2NwtUwVc0}

\end{document}
